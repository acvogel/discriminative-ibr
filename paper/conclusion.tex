We presented a discriminative model of communication that embodies the
central insights of recursive generative models of pragmatic reasoning
but is more computationally efficient, particularly at decision
time. Using experiments involving simple and complex reference games,
we showed that the model displays human-like pragmatic behavior.  In
closing, we note that the models have additional advantages that we
were unable to explore here, including (i) the ability to reason in
terms of partial, heterogeneous representations of the environment,
(ii) a decoupling of inferential power (depth of iteration) from
memory (dimensionality of the hidden representations), and (iii) a
level of computational efficiency that makes them scalable to truly 
massive problems involving language and action together.
